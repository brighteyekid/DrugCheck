\documentclass[12pt,a4paper]{report}
\usepackage[utf8]{inputenc}
\usepackage{graphicx}
\usepackage{hyperref}
\usepackage{listings}
\usepackage{xcolor}
\usepackage{geometry}
\usepackage{titlesec}
\usepackage{enumitem}
\usepackage{booktabs}
\usepackage{multicol}

\geometry{a4paper, margin=1in}

\hypersetup{
    colorlinks=true,
    linkcolor=blue,
    filecolor=magenta,
    urlcolor=blue,
}

\titleformat{\chapter}[display]
  {\normalfont\huge\bfseries}{\chaptertitlename\ \thechapter}{20pt}{\Huge}
\titlespacing*{\chapter}{0pt}{-50pt}{40pt}

\definecolor{codegreen}{rgb}{0,0.6,0}
\definecolor{codegray}{rgb}{0.5,0.5,0.5}
\definecolor{codepurple}{rgb}{0.58,0,0.82}
\definecolor{backcolour}{rgb}{0.95,0.95,0.92}

\lstdefinestyle{mystyle}{
    backgroundcolor=\color{backcolour},   
    commentstyle=\color{codegreen},
    keywordstyle=\color{magenta},
    numberstyle=\tiny\color{codegray},
    stringstyle=\color{codepurple},
    basicstyle=\ttfamily\footnotesize,
    breakatwhitespace=false,         
    breaklines=true,                 
    captionpos=b,                    
    keepspaces=true,                 
    numbers=left,                    
    numbersep=5pt,                  
    showspaces=false,                
    showstringspaces=false,
    showtabs=false,                  
    tabsize=2
}

\lstset{style=mystyle}

\begin{document}

\begin{titlepage}
\centering
{\Huge\bfseries DrugCheck\\}
\vspace{1.5cm}
{\Large\bfseries Comprehensive Health Management System\\}
\vspace{1.5cm}
{\large Technical Documentation}
\vspace{3cm}

\begin{tabular}{rl}
\Large\textbf{Version:} & \Large 1.0.0 \\
\Large\textbf{Date:} & \Large \today \\
\end{tabular}

\vfill

{\large A complete medication and health management solution}

\end{titlepage}

\tableofcontents

\chapter{Introduction}

\section{Project Overview}
DrugCheck is a comprehensive health management system designed to help users manage medications, identify drug interactions, track health data, and provide critical medical information during emergencies. The platform combines powerful medication safety features with AI-powered health insights and shareable medical IDs.

\section{Problem Statement}
Healthcare fragmentation leads to medication errors, missed interactions, and lack of emergency access to critical medical information. Studies show that:
\begin{itemize}
    \item Over 1.5 million preventable adverse drug events occur annually in the United States
    \item Drug interactions account for approximately 3-5\% of all in-hospital medication errors
    \item Emergency responders often lack critical patient information during emergencies
    \item Medication non-adherence costs the healthcare system billions annually
    \item Many patients lack knowledge of lower-cost alternatives to expensive medications
\end{itemize}

DrugCheck addresses these challenges through an integrated approach to medication and health data management.

\section{Objectives}
The primary objectives of DrugCheck are to:
\begin{itemize}
    \item Prevent adverse drug interactions through intelligent screening
    \item Provide immediate access to critical medical information in emergencies
    \item Help users find cost-effective alternatives to expensive medications
    \item Support mental health concerns related to medication use
    \item Generate personalized health insights based on user health profiles
    \item Ensure data privacy while maintaining accessibility
\end{itemize}

\chapter{System Architecture}

\section{Technology Stack}
DrugCheck is built using the following technologies:
\begin{itemize}
    \item Frontend: React.js with TypeScript
    \item State Management: React Context API
    \item Animation: Framer Motion
    \item Styling: CSS with responsive design principles
    \item QR Code Generation: qrcode.react
    \item PDF Generation: Client-side rendering
    \item Data Storage: LocalStorage for persistent data
\end{itemize}

\section{Component Architecture}
The application follows a component-based architecture with the following key components:
\begin{itemize}
    \item \textbf{DrugProvider}: Central context provider for application state
    \item \textbf{InteractionChecker}: Manages drug selection and interaction checking
    \item \textbf{MedicationReport}: Generates comprehensive PDF reports
    \item \textbf{UniversalHealthID}: Creates and manages health ID QR codes
    \item \textbf{MedicineAlternativeFinder}: Searches for medication alternatives
    \item \textbf{MentalHealthChatbot}: Provides medication-related mental health support
    \item \textbf{HealthInsightGenerator}: Analyzes health data for insights
\end{itemize}

\section{Data Flow}
\begin{enumerate}
    \item User inputs health data and medications
    \item Data is stored in the DrugContext
    \item Components access and process the data as needed
    \item AI analysis is performed client-side
    \item Results are presented to the user and can be exported/shared
\end{enumerate}

\chapter{Core Features}

\section{Medication Interaction Checker}
The Medication Interaction Checker allows users to:
\begin{itemize}
    \item Search for medications from a comprehensive database
    \item Add multiple medications to check for interactions
    \item View detailed interaction results with severity ratings
    \item Generate comprehensive PDF reports
    \item Save interaction results for future reference
\end{itemize}

\subsection{Implementation Details}
The interaction checker uses:
\begin{itemize}
    \item Drug search API with fuzzy matching
    \item Interaction severity classification algorithm
    \item PDF generation with styled templates
    \item Persistent storage of medication history
\end{itemize}

\section{Universal Health ID}
The Universal Health ID feature creates QR codes containing critical medical information:
\begin{itemize}
    \item User's personal information (name, age, blood type)
    \item Emergency contact information
    \item Medical conditions and allergies
    \item Current medications
    \item AI-generated emergency alerts
\end{itemize}

\subsection{QR Code Implementation}
\begin{itemize}
    \item Uses MECARD format for compatibility with standard QR readers
    \item Encodes all data directly within the QR code (no external database required)
    \item Includes visual formatting and emojis for better readability when scanned
    \item AI-generated emergency alerts are highlighted for first responders
\end{itemize}

\subsection{AI Analysis}
The AI component analyzes user health data to:
\begin{itemize}
    \item Identify critical conditions (diabetes, epilepsy, etc.)
    \item Detect important allergies (especially medication allergies)
    \item Recognize potential medication interactions
    \item Generate emergency recommendations for first responders
    \item Provide personalized health insights for the user
\end{itemize}

\section{Medicine Alternative Finder}
This component allows users to:
\begin{itemize}
    \item Search for a specific medication
    \item View generic alternatives and similar medications
    \item Compare costs and effectiveness
    \item Find lower-cost options for expensive medications
\end{itemize}

\section{Mental Health Support}
The Mental Health Chatbot provides:
\begin{itemize}
    \item Contextual support based on user's medication profile
    \item A secure messaging interface for documenting concerns
    \item Storage of conversation history for future reference
    \item Resources related to medication effects on mental health
\end{itemize}

\section{Health Insights Generator}
This feature:
\begin{itemize}
    \item Analyzes user health metrics
    \item Generates personalized health insights
    \item Provides recommendations for health monitoring
    \item Identifies potential health risks based on conditions and medications
\end{itemize}

\chapter{User Interface Design}

\section{Design Principles}
DrugCheck's UI follows these key principles:
\begin{itemize}
    \item Accessibility: Ensuring usability for all users
    \item Responsiveness: Adapting to all screen sizes and devices
    \item Clarity: Presenting complex medical information clearly
    \item Consistency: Maintaining uniform design patterns
    \item Visual Hierarchy: Emphasizing critical information
\end{itemize}

\section{Key UI Components}

\subsection{Navigation}
\begin{itemize}
    \item Responsive navbar with animated indicators
    \item Mobile-friendly navigation menu
    \item Context-aware navigation highlighting
\end{itemize}

\subsection{Medication Search}
\begin{itemize}
    \item Real-time search results with debouncing
    \item Clear visual presentation of medication options
    \item Recent search history functionality
\end{itemize}

\subsection{Interaction Results}
\begin{itemize}
    \item Color-coded severity indicators (severe, moderate, minor, unknown)
    \item Expandable details for in-depth information
    \item Actionable recommendations
\end{itemize}

\subsection{Health ID Display}
\begin{itemize}
    \item High-contrast QR code for optimal scanning
    \item Preview mode to verify displayed information
    \item AI insights view with clear categorization
    \item Emergency alert highlighting
\end{itemize}

\chapter{Technical Implementation}

\section{State Management}
DrugCheck uses React Context for state management:
\begin{itemize}
    \item \textbf{DrugContext}: Central state store for all application data
    \item \textbf{useDrugContext}: Custom hook for accessing the context
    \item \textbf{DrugProvider}: Provider component that wraps the application
\end{itemize}

The context maintains:
\begin{itemize}
    \item User health data
    \item Medication lists
    \item Interaction results
    \item Recent searches
    \item Mental health conversation history
\end{itemize}

\section{Data Persistence}
Data is persisted using:
\begin{itemize}
    \item LocalStorage for all user health data
    \item Session storage for temporary application state
    \item User-triggered save/export functionality for reports
\end{itemize}

\section{Offline Functionality}
DrugCheck is designed to work offline:
\begin{itemize}
    \item QR codes contain all data needed for emergency use
    \item Health ID works without internet connectivity
    \item AI analysis happens client-side
    \item Downloaded reports and QR codes are accessible without network
\end{itemize}

\section{Performance Optimization}
Performance is optimized through:
\begin{itemize}
    \item Component memoization to prevent unnecessary re-renders
    \item Lazy loading of non-critical components
    \item Efficient state updates to minimize render cycles
    \item Debounced search to reduce API calls
    \item Optimized QR code generation
\end{itemize}

\chapter{Security and Privacy}

\section{Data Storage Approach}
DrugCheck prioritizes user privacy:
\begin{itemize}
    \item All sensitive health data is stored locally on the user's device
    \item No data is transmitted to external servers without explicit user action
    \item QR codes are generated client-side
    \item User has full control over data sharing
\end{itemize}

\section{Data Sharing}
When users choose to share data:
\begin{itemize}
    \item QR codes can be downloaded and printed
    \item PDF reports can be downloaded for healthcare provider sharing
    \item Web Share API is used for direct device-to-device sharing
    \item No central database stores user health information
\end{itemize}

\chapter{Key Components Documentation}

\section{UniversalHealthID Component}
The UniversalHealthID component is responsible for:
\begin{itemize}
    \item Collecting user health information
    \item Generating QR codes with medical data
    \item Providing AI analysis of health conditions
    \item Enabling download and sharing of medical IDs
\end{itemize}

\subsection{Key Functions}

\begin{lstlisting}[language=JavaScript, caption=QR Code Generation]
const getQRCodeData = () => {
  // Create a well-formatted text that will display nicely when scanned
  // Using emojis and clear formatting to make the data more visually appealing
  
  // Add AI emergency notes if available
  let emergencyNotesText = '';
  if (aiAnalysis && aiAnalysis.emergencyNotes && aiAnalysis.emergencyNotes.length > 0) {
    emergencyNotesText = `\n⚠️ AI EMERGENCY ALERTS:\n${aiAnalysis.emergencyNotes.join('\n')}\n`;
  }
  
  const medicalData = 
`🏥 MEDICAL ID: ${formData.name} 🏥
------------------------
📋 PERSONAL INFO:
👤 Name: ${formData.name}
🔢 Age: ${formData.age}
🩸 Blood Type: ${formData.bloodType}
📞 Emergency: ${formData.emergencyContact}

⚠️ ALLERGIES:
${formData.allergies || 'None listed'}

🩺 MEDICAL CONDITIONS:
${formData.conditions || 'None listed'}

💊 MEDICATIONS:
${formData.medications || 'None listed'}${emergencyNotesText}
------------------------
⏱️ Created: ${new Date().toLocaleDateString()}
EMERGENCY MEDICAL INFORMATION
PLEASE CONTACT HEALTHCARE PROVIDER`;

  return medicalData;
};
\end{lstlisting}

\subsection{AI Analysis Implementation}

\begin{lstlisting}[language=JavaScript, caption=AI Health Analysis]
// Simulate AI analysis of medical conditions and medications
const generateAIInsights = (formData: any) => {
  const insights = [];
  const recommendations = [];
  const emergencyNotes = [];
  
  // Check for allergies
  if (formData.allergies && formData.allergies.toLowerCase().includes('penicillin')) {
    insights.push('Penicillin allergy detected - This is a common antibiotic allergy');
    recommendations.push('Alert healthcare providers about penicillin allergy before any treatment');
    emergencyNotes.push('DO NOT ADMINISTER PENICILLIN OR RELATED ANTIBIOTICS');
  }
  
  // Check for conditions
  if (formData.conditions) {
    const conditions = formData.conditions.toLowerCase();
    
    if (conditions.includes('diabetes')) {
      insights.push('Diabetes detected - Blood sugar monitoring may be necessary');
      recommendations.push('Regular blood glucose monitoring recommended');
      emergencyNotes.push('CHECK BLOOD GLUCOSE LEVELS IF UNCONSCIOUS');
    }
    
    // Additional condition checks...
  }
  
  // Check for medications
  if (formData.medications) {
    const medications = formData.medications.toLowerCase();
    
    if (medications.includes('insulin')) {
      insights.push('Insulin medication detected - Indicates diabetes management');
      recommendations.push('Carry fast-acting glucose source for hypoglycemia');
      emergencyNotes.push('RISK OF HYPOGLYCEMIA - MAY NEED GLUCOSE IF UNCONSCIOUS');
    }
    
    // Additional medication checks...
  }
  
  return {
    insights,
    recommendations,
    emergencyNotes,
    generatedAt: new Date().toISOString(),
  };
};
\end{lstlisting}

\section{MedicationReport Component}
The MedicationReport component generates comprehensive PDF reports:
\begin{itemize}
    \item Formats interaction data for printing
    \item Applies appropriate styling for print media
    \item Enables download of reports as PDF files
    \item Includes metadata and timestamp information
\end{itemize}

\subsection{PDF Generation}
\begin{lstlisting}[language=CSS, caption=PDF Styling for Reports]
@media print {
  .medication-report {
    max-width: 100%;
    margin: 0;
    padding: 20px;
    background-color: white;
    font-size: 12pt;
  }
  
  .glass-panel {
    border: 1px solid #e2e8f0;
    box-shadow: none;
    break-inside: avoid;
    margin-bottom: 20px;
    page-break-inside: avoid;
  }
  
  .section-header {
    background-color: #f8fafc;
    padding: 10px 15px;
    border-bottom: 1px solid #e2e8f0;
    font-weight: bold;
  }
  
  /* Additional print styles */
}
\end{lstlisting}

\section{DrugContext}
The DrugContext provides centralized state management:
\begin{itemize}
    \item Maintains user health data
    \item Tracks medication lists and interaction results
    \item Provides functions for data manipulation
    \item Handles persistence of user information
\end{itemize}

\subsection{Context Implementation}
\begin{lstlisting}[language=JavaScript, caption=Drug Context Provider]
export const DrugProvider: React.FC<{ children: ReactNode }> = ({ children }) => {
  const [recentSearches, setRecentSearches] = useState<Drug[]>([]);
  const [savedInteractions, setSavedInteractions] = useState<InteractionResult[]>([]);
  const [sidebarOpen, setSidebarOpen] = useState<boolean>(false);
  const [healthData, setHealthData] = useState<UserHealthData | null>(null);
  const [healthInsights, setHealthInsights] = useState<HealthInsight | null>(null);
  const [messages, setMessages] = useState<Message[]>([]);
  const [medications, setMedications] = useState<Drug[]>([]);
  const [interactions, setInteractions] = useState<InteractionResult[]>([]);

  // Context functions
  const addToRecentSearches = (drug: Drug) => {
    setRecentSearches(prev => {
      const newSearches = [drug, ...prev.filter(d => d.id !== drug.id)].slice(0, 10);
      return newSearches;
    });
  };

  // Additional context functions...

  return (
    <DrugContext.Provider value={{
      recentSearches,
      addToRecentSearches,
      clearRecentSearches,
      savedInteractions,
      saveInteraction,
      // Additional context values...
    }}>
      {children}
    </DrugContext.Provider>
  );
};
\end{lstlisting}

\chapter{Future Enhancements}

\section{Planned Features}
Future versions of DrugCheck will include:
\begin{itemize}
    \item Integration with wearable health devices
    \item Medication reminder and adherence tracking
    \item Advanced AI-powered health predictions
    \item Support for family/caregiver health management
    \item Expanded alternative medicine database
    \item Integration with telehealth services
    \item Personalized drug safety scores
\end{itemize}

\section{Technical Improvements}
Planned technical enhancements include:
\begin{itemize}
    \item Progressive Web App (PWA) implementation for better offline capabilities
    \item Advanced QR code encryption for additional security
    \item Machine learning for more accurate interaction predictions
    \item Blockchain-based verification of medical credentials
    \item Voice-activated emergency information access
    \item Automated medication reconciliation
\end{itemize}

\chapter{Conclusion}

DrugCheck addresses critical challenges in healthcare management by providing users with tools to:
\begin{itemize}
    \item Prevent dangerous medication interactions
    \item Provide emergency access to medical information
    \item Find cost-effective alternatives to expensive medications
    \item Monitor mental health in relation to medications
    \item Generate personalized health insights
\end{itemize}

By putting control of health data in users' hands while ensuring accessibility during emergencies, DrugCheck represents a significant step forward in patient-centered healthcare management.

The platform's focus on privacy, offline functionality, and comprehensive health information makes it a valuable tool for users of all ages and health statuses, with particular benefits for those managing multiple medications or chronic conditions.

\end{document}
